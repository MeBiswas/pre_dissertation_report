% !TeX root = ../main.tex

\section{Data Analysis and Interpretation}

    \subsection{Tables and figures}
    \begin{itemize}
        \item Model-by-model performance table: model, balancing strategy, FS method (if any), AUC, F1, recall, precision, MCC, training time.
        \item ROC curves and precision-recall curves for top-performing models.
        \item Feature-importance charts for tree-based models; SHAP summary plots for interpretability.
        \item Cross-dataset generalization results: how models trained on European data perform on the second dataset, with notes on the need for retraining vs. fine-tuning.
    \end{itemize}

    \begin{table}[H]
        \centering
        \begin{tabular}{|c|c|c|}
            \hline
            \textbf{Model} & \textbf{Accuracy (\%)} & \textbf{F1 Score} \\
            \hline
            Random Forest & 92.5 & 0.89 \\
            Support Vector Machine & 90.3 & 0.85 \\
            Logistic Regression & 88.7 & 0.82 \\
            Multi-Layer Perceptron & 91.2 & 0.87 \\
            Gaussian Mixture Model & 85.4 & 0.80 \\
            \hline
        \end{tabular}
        \caption{Performance Metrics of Different Models}
        \label{tab:model_performance}
    \end{table}

    \begin{figure}[H]
        \centering
            \begin{figure}[H]
                \centering
                \begin{tikzpicture}[
                    node distance=1.8cm,
                    every node/.style={font=\small},
                    process/.style={rectangle, rounded corners, minimum width=4cm, minimum height=1cm, draw=black, thick, fill=gray!5, align=center},
                    arrow/.style={thick,->,>=stealth}
                ]
                
                    % Nodes
                    \node (input) [process] {Raw Dataset\\(Backblaze SMART Data)};
                    \node (clean) [process, below=of input] {Data Cleaning \& Preprocessing\\(Missing values, normalization)};
                    \node (gan) [process, below=of clean] {Hybrid GAN\\Synthetic Data Generation};
                    \node (feat) [process, below=of gan] {Teacher–Student\\Feature Selection};
                    \node (models) [process, below=of feat] {ML/DL Models\\(RF, SVM, LR, MLP, GMM, etc.)};
                    \node (eval) [process, below=of models] {Model Evaluation\\(Accuracy, Precision, Recall, F1, PR Curve)};
                    \node (out) [process, below=of eval] {Final Output\\(Selected Model \& Predictions)};
                    
                    % Arrows
                    \draw [arrow] (input) -- (clean);
                    \draw [arrow] (clean) -- (gan);
                    \draw [arrow] (gan) -- (feat);
                    \draw [arrow] (feat) -- (models);
                    \draw [arrow] (models) -- (eval);
                    \draw [arrow] (eval) -- (out);
                
                \end{tikzpicture}
                
                \caption{End-to-End Workflow of the Hard Drive Failure Prediction System}
            \end{figure}
        \caption{Model Accuracy}
    \end{figure}

    \subsection{Discussion points}
    \begin{itemize}
        \item Which models consistently perform well across datasets and balancing methods?
        \item Does feature selection consistently improve generalization and interpretability?
        \item Do AutoML results keep pace with manually tuned top performers?
        \item What are the trade-offs between accuracy and latency for potential real-time deployment?
        \item How robust are the results to distribution shifts and data-scarcity scenarios?
    \end{itemize}