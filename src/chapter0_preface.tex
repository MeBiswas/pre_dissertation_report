% !TeX root = ../main.tex

\section*{Preface}
\addcontentsline{toc}{section}{Preface}

\begin{justify}
    Digital payment systems and online transactions have transformed how we handle money in our daily lives, making financial activities more convenient than ever before. However, this convenience comes with a serious downside: criminals have found new ways to exploit these systems, causing massive financial losses to both individuals and financial institutions. Credit card fraud alone costs the world over \$32 billion annually, making it one of the most damaging forms of financial crime today.

    Traditional methods of detecting fraud typically depend on fixed rules and predefined patterns created by experts. While these approaches worked in the past, they cannot keep pace with modern fraudsters who constantly adapt their tactics. This challenge has created an urgent need for smarter detection methods. Machine learning and deep learning offer promising solutions because they can identify complex fraud patterns, adapt to emerging threats, and process enormous volumes of transaction data in real-time.
    
    Even small improvements in fraud detection—such as catching a few more fraudulent transactions or reducing false alarms—can save millions of dollars and significantly improve customer satisfaction. This makes credit card fraud detection a valuable application of machine learning technology.

    This report presents a systematic investigation into credit card fraud detection using machine learning techniques. The research focuses on three critical challenges: handling imbalanced datasets (where fraudulent transactions are much rarer than legitimate ones), selecting the most relevant features for detection, and creating automated detection systems. The objective is to develop a reliable methodology that can evaluate different algorithms using both real-world and synthetic datasets, identify effective practices, and propose practical solutions for real-time fraud detection. This work builds upon recent research in traditional machine learning, deep learning, hybrid models, data balancing techniques, and feature engineering.
\end{justify}