% !TeX root = ../main.tex

\begin{justify}
    Credit card fraud has slowly turned into one of those problems that just keeps growing no matter what people try. Both banks and regular customers end up losing a lot of money every year, and the people behind these scams constantly change the way they operate. With almost everything moving toward online payments now, the tricks have only gotten more complicated, and the older, rule-based systems that used to catch fraud simply aren't sharp enough anymore.

    Because the situation keeps shifting, newer approaches based on machine learning and deep learning have started to feel more practical. They don’t rely on rigid rules. Instead, they notice odd patterns, adjust when scammers try something new, and sort through huge piles of transaction data without slowing down. In this study, I looked at several models—Logistic Regression, Naive Bayes, Decision Trees, Random Forests, XGBoost, LightGBM, SVM, and KNN—to get a sense of how each one behaves when trying to detect suspicious transactions.

    But the models are only part of the story. A lot of the effort actually goes into preparing the data. This includes creating only a small time-based or amount-related features, figuring out which variables truly matters, removing samples that add noise, and applying scaling techniques e.g. Robust Scaling or Min-Max Scaling. Since fraud data is extremely unbalanced, I also used methods like SMOTE, GAN-generated samples, and careful cross-validation to avoid the model being biased toward the larger class.

    In the end, the whole idea behind this work is to build a practical path for developing fraud-detection systems that lenders can actually trust and scale. The study mixes different techniques and shows how they can work together to build models that respond quickly enough for real-world, real-time financial systems.
\end{justify}