% !TeX root = ../main.tex

\addcontentsline{toc}{section}{Preface}

\begin{justify}
    Credit card fraud has become a major global concern, resulting in billions of dollars in annual financial losses and posing serious risks to both individuals and financial institutions. As digital payments grow, fraudsters continually adapt, making traditional rule-based detection methods insufficient for modern threats.

    \acrfull{ml} and \acrfull{dl} offer powerful alternatives by identifying hidden fraud patterns, adapting to new behaviours, and analyzing large transaction streams in real time. In this study, a wide range of \acrshort{ml} models—\acrlong{lr}, \acrlong{nb}, \acrlong{dt}, \acrlong{rf}, \acrlong{xgboost}, \acrlong{lgbm}, \acrlong{svm}, and \acrlong{knn}—are explored to evaluate their effectiveness in fraud detection.

    The research also emphasizes essential data-centric components such as feature engineering for time and amount variables, feature selection methods, prototype selection, and scaling techniques like Robust Scaler and Min-Max Scaler. To handle the extreme class imbalance inherent in fraud datasets, strategies including \acrshort{smote}, \acrshort{gan}-based augmentation, and careful cross-validation are employed.

    This report presents a systematic and practical exploration of these techniques to build reliable, scalable, and real-time fraud detection models suited for modern financial systems.
\end{justify}