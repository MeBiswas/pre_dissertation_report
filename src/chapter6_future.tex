% !TeX root = ../main.tex

The field of study presented in this chapter holds significant potential for future research and development. Several avenues can be explored to enhance the current understanding and application of the concepts discussed. One promising direction is the integration of emerging technologies, such as artificial intelligence and machine learning, to improve data analysis and predictive modeling. Additionally, further investigation into the scalability of the proposed methods could lead to broader applicability across various industries. Another area worth exploring is the development of more robust algorithms that can handle larger datasets and more complex scenarios. Collaborative efforts between academia and industry could also foster innovation and lead to practical implementations of the theoretical frameworks presented. Overall, the future scope of this field is vast, and continued research will undoubtedly yield valuable insights and advancements.
Moreover, interdisciplinary approaches that combine insights from related fields could provide a more holistic understanding of the challenges and opportunities in this area. For instance, incorporating perspectives from sociology, economics, and environmental science could lead to more comprehensive solutions that address not only technical aspects but also societal impacts. Finally, there is a need for ongoing evaluation and refinement of existing models to ensure they remain relevant in the face of rapidly changing technological landscapes. By pursuing these future directions, researchers can contribute to the evolution of the field and its practical applications.
In conclusion, the future scope of this chapter's subject matter is rich with possibilities. By embracing
\begin{itemize}
    \item Extensions to real-time and streaming fraud detection with low latency, including edge inference and online learning.
    \item Deeper exploration of deep generative oversampling in varied domains and under different imbalance regimes, with robust evaluation of diversity and realism of synthetic samples.
    \item AutoML-driven experiment replication across more diverse datasets and more complex feature spaces, with careful emphasis on explainability outputs.
    \item Privacy-preserving and federated learning approaches to enable cross-institution fraud detection without sharing raw data; benchmark privacy-utility trade-offs.
    \item Explainable AI (XAI) enhancements to improve stakeholder trust and regulatory compliance, including visual explanations of decisions and feature contributions.
    \item Cross-domain generalization studies to assess how well fraud-detection models trained on one country's data transfer to another country’s data and what adaptations are necessary.
\end{itemize}