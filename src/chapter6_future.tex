% !TeX root = ../main.tex

While this study provides a detailed evaluation of multiple machine learning models for credit card 
fraud detection, several opportunities remain for enhancing model robustness, scalability, and 
real-world applicability. This chapter highlights the key directions for future research based on the 
observations, limitations, and insights drawn from the experiments.

\section*{Enhancing Data Quality and Diversity}

A major limitation in fraud detection research is the reliance on a small number of publicly available 
datasets, such as the European Credit Card Fraud dataset. Although this dataset reflects real-world 
class imbalance, it is temporally narrow and anonymized through PCA transformations. Future work can 
expand this research by:

    \begin{itemize}
        \item Incorporating additional datasets from different banks, regions, and time periods to 
        capture diverse fraud patterns.
        \item Exploring federated or privacy-preserving data-sharing frameworks to support collaborative 
        fraud detection across institutions without compromising sensitive customer information.
        \item Collecting richer merchant-level, device-level, and customer-behavioural attributes to 
        improve contextual understanding of fraud.
    \end{itemize}

\section*{Advanced Data Balancing and Augmentation}

This study highlighted the challenges posed by synthetic oversampling techniques such as SMOTE and GAN. 
While these methods increased minority sample volume, they often caused distribution drift and severe 
overfitting. Future work may focus on:

    \begin{itemize}
        \item Developing \textbf{generative models constrained by real fraud manifolds}, ensuring 
        synthetic samples remain close to the authentic minority distribution.
        \item Exploring \textbf{diffusion models}, \textbf{variational autoencoders (VAEs)}, or 
        \textbf{conditional GANs} tailored for tabular anomaly data.
        \item Adopting \textbf{hybrid resampling techniques} that combine undersampling, clustering, and 
        density estimation to minimize synthetic noise.
    \end{itemize}

\section*{Temporal and Streaming Fraud Detection}

Fraud patterns evolve rapidly due to adversarial adaptation, seasonal behaviour, and emerging attack 
vectors. The static models trained in this study assume that the data distribution remains stable over 
time. Future work should investigate:

    \begin{itemize}
        \item \textbf{Online learning and streaming algorithms} capable of updating fraud patterns 
        continuously as new transactions arrive.
        \item \textbf{Temporal cross-validation and sliding-window training} to better capture concept drift.
        \item \textbf{Drift-detection mechanisms} that automatically trigger model retraining or 
        recalibration when performance declines.
    \end{itemize}

\section*{Cost-Sensitive and Risk-Aware Modelling}

The financial impact of false positives and false negatives is not symmetric. Missing a fraud event is 
far more costly than flagging a legitimate transaction. This study used conventional classification 
metrics, but real-world fraud detection requires:

    \begin{itemize}
        \item \textbf{Cost-sensitive learning frameworks} that incorporate transaction value, customer risk, 
        and business impact.
        \item \textbf{Custom loss functions} that weigh minority misclassification more heavily.
        \item \textbf{Threshold-optimization strategies}, such as maximizing expected financial savings rather 
        than F1-score.
    \end{itemize}

\section*{Explainability and Regulatory Compliance}

Financial institutions require transparent and explainable models to comply with auditing standards and 
customer transparency regulations. Although ensemble models such as XGBoost and LightGBM were among the 
best performers in this study, their complexity poses challenges.

Future research may focus on:

    \begin{itemize}
        \item Integrating \textbf{explainable AI (XAI)} tools such as SHAP, LIME, and counterfactual reasoning 
        to make model predictions interpretable.
        \item Designing \textbf{human-in-the-loop fraud detection systems} where fraud analysts interact with 
        model outputs and provide feedback.
        \item Studying \textbf{explainability-performance trade-offs} across different model families.
    \end{itemize}

\section*{Model Deployment and Real-Time Constraints}

Fraud detection often requires decisions to be made in milliseconds. Although this study focused on 
model performance, real-world deployment introduces new constraints:

    \begin{itemize}
        \item \textbf{Latency optimization} for serving models in production-grade environments.
        \item \textbf{Scalability of ensemble methods} when processing millions of daily transactions.
        \item \textbf{Robust monitoring pipelines} for drift detection, retraining alerts, and anomaly 
        detection at scale.
    \end{itemize}

\section*{Hybrid and Multi-Model Architectures}

Based on this study, individual models excel in different areas—LightGBM provides high precision, 
XGBoost balances recall and precision, and Random Forest offers robustness. Future systems can combine 
these strengths by:

    \begin{itemize}
        \item Building \textbf{stacked or blended ensemble architectures} that fuse outputs from multiple 
        high-performing models.
        \item Investigating \textbf{multi-stage fraud detection pipelines}, where lightweight linear models 
        filter transactions before more complex models perform final classification.
        \item Applying \textbf{graph neural networks (GNNs)}, which model customer–merchant relationships 
        and transaction networks for enhanced fraud pattern recognition.
    \end{itemize}

\section*{Future Research Summary}

The findings from this research demonstrate that fraud detection remains a challenging and evolving 
problem requiring continuous innovation. The most promising directions include leveraging advanced 
generative models, deploying drift-aware architectures, integrating cost-sensitive learning, enhancing 
interpretability, and building scalable real-time systems.

As digital financial systems continue to expand, fraudsters will adapt their strategies, making 
adaptive, transparent, and robust AI-driven fraud detection essential for protecting consumers and 
institutions in the coming years.