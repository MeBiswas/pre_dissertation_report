% !TeX root = ../main.tex

\section{Research Methodology}

\subsection{Research questions}
    \begin{itemize}
        \item Which ML/DL models deliver the strongest fraud-detection performance on imbalanced European credit-card datasets \cite{kaggle_dataset}, and how do they compare when augmented by different balancing and feature-selection strategies?
        \item How does feature selection (e.g., LDA, PCA, GA-based selectors) impact model performance and interpretability?
    \end{itemize}

\subsection{Data sources}
    European credit card dataset \cite{kaggle_dataset} (the standard 30 features V1–V28, Time, Amount; fraud label). Use the original data with preprocessing as a baseline, and also consider a SMOTE/ADASYN-augmented version for comparison.

\subsection{Experimental design}
    \begin{description}
        \item \textbf{Baseline models:} Logistic Regression (\acrshort{lr}), Decision Tree (\acrshort{dt}), Random Forest (\acrshort{rf}), XGBoost (\acrshort{xgboost}), SVM (\acrshort{svm}), K-Nearest Neighbors (\acrshort{knn}), Gaussian Naive Bayes (\acrshort{gaussiannb}) as a \acrlong{ml} baseline.
        \item \textbf{Feature handling:} Standardize features.
        \item \textbf{Balancing strategies:}
            \begin{itemize}
                \item No resampling (baseline on original data)
                \item Classic oversampling (SMOTE, ADASYN)
                \item Deep generative oversampling (GAN)
                \item Undersampling methods as a complement for comparison (careful handling to avoid loss of information)
            \end{itemize}
        \item \textbf{Model optimization:}
            \begin{itemize}
                \item Hyperparameter tuning for each model (regularization, tree depth, learning rate, number of trees, kernel parameters, etc.)
                \item Cross-validation (e.g., stratified k-fold) to ensure robustness.
            \end{itemize}
        \item \textbf{Evaluation metrics:}
            \begin{itemize}
                \item \textbf{Primary:} AUC/ROC, F1-score, recall (fraud detection emphasis), precision
                \item \textbf{Secondary:} Accuracy, PR-AUC
                \item Computational metrics: training time, inference latency (for real-time considerations)
            \end{itemize}
        \item \textbf{Validation strategy:}
            \begin{itemize}
                \item Stratified k-fold cross-validation for robust estimates
                \item Holdout test set for final evaluation
                \item Statistical significance testing (e.g., paired t-tests or non-parametric tests) to compare top models
            \end{itemize}
        \item \textbf{Reproducibility:} Document seeds, data preprocessing steps, and exact hyperparameters; consider releasing code and configuration files.
        \item \textbf{Data analysis plan:}
            \begin{itemize}
                \item Compare performance across models and balancing strategies using aggregated metrics (mean and 95\% CIs) over cross-validation folds.
                \item Analyze feature importance and model explainability (SHAP values or feature-importance plots for tree ensembles; LIME for select models).
                \item Investigate the impact of feature selection by comparing performance with and without FS, and by analyzing the most informative features across models.
                \item Evaluate robustness to distribution shifts by testing models trained on European data on the second dataset without retraining, then with retraining.
            \end{itemize}
        \item \textbf{Deliverables:}
            \begin{itemize}
                \item A comprehensive experimental report detailing methodology, results, and interpretations.
                \item Clear figures and tables showing model performance across configurations.
                \item An Abbreviations section and a Findings/Recommendations section suitable for the dissertation narrative.
            \end{itemize}
        \item \textbf{Tools and environment :}
            \begin{itemize}
                \item Python-based ML stack (scikit-learn for baseline models; XGBoost or LightGBM for gradient-boosted trees; TensorFlow/Keras or PyTorch for DL baselines; imbalanced-learn for SMOTE/ADASYN).
                \item \textbf{Data visualization:} matplotlib/seaborn for performance plots; SHAP for interpretability.
                \item \textbf{Reproducibility:} Jupyter/Notebook pipelines with requirements.txt; a reproducible workflow (e.g., Makefile or a simple Python script with config files).
                \item Version control (Git) for code management.
            \end{itemize}
    \end{description}